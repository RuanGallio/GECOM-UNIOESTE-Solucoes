\documentclass[a4paper,11pt]{report}
\usepackage{latexsym,amssymb,amsfonts,amsmath}
%\usepackage[brazilian]{babel}
% \usepackage[ansinew]{inputenc}
%\usepackage{tabularx}
\usepackage{color}
\usepackage{enumitem}
%\usepackage{multicol}

% Configuração do tamanho da página e das margens
\usepackage[a4paper,top=2.5cm,bottom=2cm,left=2.5cm,right=2cm]{geometry}

% Configuração dos parágrafos e do espaçamento
\setlength{\parindent}{20mm}         % Tabulação da primeira linha em cada parágrafo
\setlength{\parskip}{0.5\baselineskip} %Distância entre parágrafos
\renewcommand{\baselinestretch}{1.3}   % Espaçamento do texto (simples, duplo, 1.5, etc...)


\pagenumbering{arabic}


\newcounter{execount}  \setcounter{execount}{1}
\newcommand{\newexe}{ \vspace{\baselineskip} \hrule  \vspace{.5\baselineskip} \noindent {\bfseries
\arabic{execount}.} \addtocounter{execount}{1} }


\DeclareMathOperator{\sen}{sen} %
\DeclareMathOperator{\arcsen}{arcsen}



\allowdisplaybreaks

\newenvironment{solucao}{ \vspace{.5\baselineskip} \renewcommand{\baselinestretch}{1}
                           \noindent {\bfseries Solução:} }
                        { \vspace{.5\baselineskip} \renewcommand{\baselinestretch}{1.2}
                           \hfill {\small $\blacksquare$} }

\pagestyle{empty}

\begin{document}



\centerline{ {\Large Solução da 44 OBM (2022)} }

\vspace{\baselineskip}


\newexe Dado $0 < a < 1$, determine todas as funções f: $\mathbb{R} \to \mathbb{R}$ contínuas em $x=0$
tais que $f(x) + f(ax) = x, \forall x\in \mathbb{R}$


\newexe Considere o conjunto $G$ de matrizes $2 \times 2$ dado por
\[
    G = \left\{
    \begin{pmatrix}
        a & b \\
        c & d
    \end{pmatrix} | a, b, c, d \in \mathbb{Z}, ad - bc = 1, c \text{ é múltiplo de 3}
    \right\}
\]
e as matrizes em $G$
\[
    A = \begin{pmatrix}
        1 & 1 \\
        0 & 1
    \end{pmatrix}
    \quad B = \begin{pmatrix}
        -1 & 1 \\
        -3 & 2
    \end{pmatrix}.
\]
Mostre que qualquer matriz de $G$ pode ser escrita como por um produto $M_1, M_2,\dots, M_r$ com
$M_i \in \{A, A^{-1}, B, B^{-1}\}$, $\forall i \leq r$.


\newexe Seja $(a_{n})_{n\in\mathbb{N}}$ uma sequência de inteiros. Definimos $a^{(0)}_{n} = a_{n}$, para todo $n$ natural. Para todo
inteiro$ M \geq 0$, definimos $(a^{(M+1)}_{n})_{n\in\mathbb{N}}: a^{(M)}_{n+1} - a^{(M)}_{n}, \forall n \in \mathbb{N}$. E dizemos que $(a_{n})_{n\in\mathbb{N}}$
é $(M+1)$-autorreferente se existem $k_1$ e $k_2$ naturais fixados, tais que $a_{n+k_{1}}=a^{(M+1)}_{n+k_{2}}, \forall n \in \mathbb{N} $.
\begin{enumerate}[label=\alph*)]
    \item Existe uma sequência de inteiros tal que o menor $M$ para o qual ela é$ M$-autorreferente é
          $M = 2022$?

    \item Existe uma sequência estritamente crescente de inteiros positivos tal que o menor $M$ para o qual
          ela é $M$-autorreferente é $M = 2022$?
\end{enumerate}


\newexe Dados $c,a > 0$, considere a sequência $(x_{n})_{n \geq 1}$ definida por $x_1 = c$ e $x_{n+1} = x_{n}e^{-x^{a}_{n}}$ para $n \geq 1$.
Para quais valores reais de $\beta$ a série $\sum_{n=1}^{\infty} x^{\beta}_{n}$ é convergente?

\begin{solucao}
    Primeiramente, notemos que, como $e^{r}>0, \forall r \in \mathbb{R}$,  $x_{n} > 0$ para todo $n \geq 1$. Logo, $x^{a}_{n}$ também é positivo para $n \geq 1$, donde $-x^{a}_{n} < 0$.

    Agora, para analizar a convergência, aplicamos o teste da razão, isto é
    \[
        \lim_{n\to\infty} \frac{x^{\beta}_{n+1}}{x^{\beta}_{n}} = \lim_{n\to\infty} (\frac{x_{n+1}}{x_{n}})^{\beta},
    \]
    onde, substituindo $x_{n+1}$ por sua relação de recorrência, temos
    \[
        \lim_{n\to\infty} (\frac{x_{n}e^{-x^{a}_{n}}}{x_{n}})^{\beta} =  \lim_{n\to\infty} (e^{-x^{a}_{n}})^{\beta} = \lim_{n\to\infty} e^{-\beta x^{a}_{n}}.
    \]
    Para uma série convergir, é necessário que o limite acima seja menor que 1. Assim, temos que $e^{-\beta x^{a}_{n}} < 1$, donde $-\beta x^{a}_{n} < 0$, ou seja, $\beta > 0$.
    Finalmente, para $\beta = 0$, note que a série seria a soma de termos constantes iguais a 1, o que diverge. Logo, a série converge para todo $\beta \in \mathbb{R}$ tal que $\beta > 0$.

\end{solucao}

\newexe Dado $X \subset\mathbb{N}$, definimos $d(X)$ como sendo o maior $c\in[0,1]$ tal que, para quaisquer $a<c$ e $n_0 \in \mathbb{N}$,
existem $m,r\in\mathbb{N}$ com $r > n_0$ e $|X \cap [m, m+r)|/r\geq a$. Sejam $E, F \subset \mathbb{N}$ com $d(E)d(F) > 1/4$.
Prove que, para qualquer $p$ primo e $k \in \mathbb{N}$, existem $m \in E$ e $n \in F$ com $m \equiv n \pmod{p^k}$.


\newexe Seja $p \equiv \pmod{4}$ um número primo, e seja $\theta$ um ângulo tal que $\tan(\theta)$ é racional. Prove que
$\tan((p+1)\theta)$ é um número racional cujo númerador é múltiplo de $p$, ou seja $\tan((p+1)\theta) = \frac{u}{v}$ com
$u,v\in\mathbb{Z}, v>0$, mdc$(u,v) = 1$ e $u \equiv 0 \pmod{p}$.


\end{document}
